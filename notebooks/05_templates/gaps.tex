\documentclass[11pt,a4paper]{article}

\usepackage[margin=1.8cm]{geometry}
\usepackage{enumitem}
\usepackage{hyperref}
\usepackage{xcolor}

\setlength{\parindent}{0pt}
\setlist[itemize]{noitemsep, topsep=0pt}

\begin{document}

\begin{center}
    {\LARGE \textbf{Gap Analysis}}\\
    \vspace{0.2cm}
    {\large ((( company_name ))) — ((( job_title )))}\\
    \vspace{0.1cm}
    {\small Gerado em ((( generation_date )))}
\end{center}

\vspace{0.5cm}

\section*{Resumo}

Este documento lista os principais requisitos da vaga que o candidato
ainda não atende plenamente, juntamente com sugestões objetivas
para fechar cada lacuna.

\vspace{0.4cm}

((* if gaps *))
\section*{Gaps Identificados}

((* for gap in gaps *))
\subsection*{((( loop.index ))). ((( gap.title )))}

\textbf{Descrição:}\\
((( gap.description )))

\vspace{0.15cm}

\textbf{Por que isso importa para a vaga:}\\
((( gap.impact )))

\vspace{0.15cm}

\textbf{Como fechar esse gap:}
\begin{itemize}
((* for action in gap.actions *))
    \item ((( action )))
((* endfor *))
\end{itemize}

\vspace{0.3cm}
((* endfor *))
((* endif *))

\section*{Observações Finais}

Os gaps acima não representam impedimentos absolutos, mas sim
oportunidades claras de desenvolvimento alinhadas à vaga analisada.
A maioria pode ser mitigada com estudo direcionado e experiência prática
em curto ou médio prazo.

\end{document}
